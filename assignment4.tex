\documentclass[option]{article}
\begin{document}
\section{A LITERATURE REVIEW ABOUT GOOGLE MAPS}
Google maps is a web mapping service develop by Google. It offers street  maps, real time traffic conditions and route planning.The application was first announced on Google blog february,8,2005 \cite{r1}. It has advanced through serveral development version with its first design made by Lars and Lons Eilstrup at Sydney-based company. By then it featured as a desktop program but through various redesigns it is currently supported on multiple platform including mobile devices.
  are cited in this paper.\\\\
Google Maps' satellite view is a "top-down" or "birds eye" view; most of the high-resolution imagery of cities is aerial photography taken from aircraft flying at 800 to 1,500 feet (240 to 460 m), while most other imagery is from satellites. Much of the available satellite imagery is no more than three years old and is updated on a regular basis. Google Maps uses a close variant of the Mercator projection, and therefore cannot accurately show areas around the poles.\\\\
 Google Maps offers an API that allows maps to be embedded on third-party websites \cite{r2} and offers a locator for urban businesses and other organizations in numerous countries around the world.\\\\
Google Maps is available as a mobile app for the Android and iOS mobile operating systems.
The Android app was first released in September 2008, \cite{r3} though the GPS-localization feature had been in testing on cellphones since 2007. Google Maps was Apple's solution for its mapping service on iOS until the release of iOS 6 in September 2012, at which point it was replaced by Apple Maps, with Google releasing its own Google Maps standalone app on the iOS platform the following December.\\\\
Google Maps is great for just getting around \cite{r4}.It is still the best overall map \cite{r5}.Maps has all sorts of powerful features and time-saving shortcuts
\bibliographystyle{IEEEtran}
\bibliography{references}
\end{document}
